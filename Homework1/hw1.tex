\PassOptionsToPackage{unicode=true}{hyperref} % options for packages loaded elsewhere
\PassOptionsToPackage{hyphens}{url}
%
\documentclass[
]{article}
\usepackage{lmodern}
\usepackage{amssymb,amsmath}
\usepackage{ifxetex,ifluatex}
\ifnum 0\ifxetex 1\fi\ifluatex 1\fi=0 % if pdftex
  \usepackage[T1]{fontenc}
  \usepackage[utf8]{inputenc}
  \usepackage{textcomp} % provides euro and other symbols
\else % if luatex or xelatex
  \usepackage{unicode-math}
  \defaultfontfeatures{Scale=MatchLowercase}
  \defaultfontfeatures[\rmfamily]{Ligatures=TeX,Scale=1}
\fi
% use upquote if available, for straight quotes in verbatim environments
\IfFileExists{upquote.sty}{\usepackage{upquote}}{}
\IfFileExists{microtype.sty}{% use microtype if available
  \usepackage[]{microtype}
  \UseMicrotypeSet[protrusion]{basicmath} % disable protrusion for tt fonts
}{}
\makeatletter
\@ifundefined{KOMAClassName}{% if non-KOMA class
  \IfFileExists{parskip.sty}{%
    \usepackage{parskip}
  }{% else
    \setlength{\parindent}{0pt}
    \setlength{\parskip}{6pt plus 2pt minus 1pt}}
}{% if KOMA class
  \KOMAoptions{parskip=half}}
\makeatother
\usepackage{xcolor}
\IfFileExists{xurl.sty}{\usepackage{xurl}}{} % add URL line breaks if available
\IfFileExists{bookmark.sty}{\usepackage{bookmark}}{\usepackage{hyperref}}
\hypersetup{
  pdftitle={Tarea 1},
  pdfauthor={Pablo,Sofía,Román},
  pdfborder={0 0 0},
  breaklinks=true}
\urlstyle{same}  % don't use monospace font for urls
\usepackage[margin=1in]{geometry}
\usepackage{color}
\usepackage{fancyvrb}
\newcommand{\VerbBar}{|}
\newcommand{\VERB}{\Verb[commandchars=\\\{\}]}
\DefineVerbatimEnvironment{Highlighting}{Verbatim}{commandchars=\\\{\}}
% Add ',fontsize=\small' for more characters per line
\usepackage{framed}
\definecolor{shadecolor}{RGB}{248,248,248}
\newenvironment{Shaded}{\begin{snugshade}}{\end{snugshade}}
\newcommand{\AlertTok}[1]{\textcolor[rgb]{0.94,0.16,0.16}{#1}}
\newcommand{\AnnotationTok}[1]{\textcolor[rgb]{0.56,0.35,0.01}{\textbf{\textit{#1}}}}
\newcommand{\AttributeTok}[1]{\textcolor[rgb]{0.77,0.63,0.00}{#1}}
\newcommand{\BaseNTok}[1]{\textcolor[rgb]{0.00,0.00,0.81}{#1}}
\newcommand{\BuiltInTok}[1]{#1}
\newcommand{\CharTok}[1]{\textcolor[rgb]{0.31,0.60,0.02}{#1}}
\newcommand{\CommentTok}[1]{\textcolor[rgb]{0.56,0.35,0.01}{\textit{#1}}}
\newcommand{\CommentVarTok}[1]{\textcolor[rgb]{0.56,0.35,0.01}{\textbf{\textit{#1}}}}
\newcommand{\ConstantTok}[1]{\textcolor[rgb]{0.00,0.00,0.00}{#1}}
\newcommand{\ControlFlowTok}[1]{\textcolor[rgb]{0.13,0.29,0.53}{\textbf{#1}}}
\newcommand{\DataTypeTok}[1]{\textcolor[rgb]{0.13,0.29,0.53}{#1}}
\newcommand{\DecValTok}[1]{\textcolor[rgb]{0.00,0.00,0.81}{#1}}
\newcommand{\DocumentationTok}[1]{\textcolor[rgb]{0.56,0.35,0.01}{\textbf{\textit{#1}}}}
\newcommand{\ErrorTok}[1]{\textcolor[rgb]{0.64,0.00,0.00}{\textbf{#1}}}
\newcommand{\ExtensionTok}[1]{#1}
\newcommand{\FloatTok}[1]{\textcolor[rgb]{0.00,0.00,0.81}{#1}}
\newcommand{\FunctionTok}[1]{\textcolor[rgb]{0.00,0.00,0.00}{#1}}
\newcommand{\ImportTok}[1]{#1}
\newcommand{\InformationTok}[1]{\textcolor[rgb]{0.56,0.35,0.01}{\textbf{\textit{#1}}}}
\newcommand{\KeywordTok}[1]{\textcolor[rgb]{0.13,0.29,0.53}{\textbf{#1}}}
\newcommand{\NormalTok}[1]{#1}
\newcommand{\OperatorTok}[1]{\textcolor[rgb]{0.81,0.36,0.00}{\textbf{#1}}}
\newcommand{\OtherTok}[1]{\textcolor[rgb]{0.56,0.35,0.01}{#1}}
\newcommand{\PreprocessorTok}[1]{\textcolor[rgb]{0.56,0.35,0.01}{\textit{#1}}}
\newcommand{\RegionMarkerTok}[1]{#1}
\newcommand{\SpecialCharTok}[1]{\textcolor[rgb]{0.00,0.00,0.00}{#1}}
\newcommand{\SpecialStringTok}[1]{\textcolor[rgb]{0.31,0.60,0.02}{#1}}
\newcommand{\StringTok}[1]{\textcolor[rgb]{0.31,0.60,0.02}{#1}}
\newcommand{\VariableTok}[1]{\textcolor[rgb]{0.00,0.00,0.00}{#1}}
\newcommand{\VerbatimStringTok}[1]{\textcolor[rgb]{0.31,0.60,0.02}{#1}}
\newcommand{\WarningTok}[1]{\textcolor[rgb]{0.56,0.35,0.01}{\textbf{\textit{#1}}}}
\usepackage{graphicx,grffile}
\makeatletter
\def\maxwidth{\ifdim\Gin@nat@width>\linewidth\linewidth\else\Gin@nat@width\fi}
\def\maxheight{\ifdim\Gin@nat@height>\textheight\textheight\else\Gin@nat@height\fi}
\makeatother
% Scale images if necessary, so that they will not overflow the page
% margins by default, and it is still possible to overwrite the defaults
% using explicit options in \includegraphics[width, height, ...]{}
\setkeys{Gin}{width=\maxwidth,height=\maxheight,keepaspectratio}
\setlength{\emergencystretch}{3em}  % prevent overfull lines
\providecommand{\tightlist}{%
  \setlength{\itemsep}{0pt}\setlength{\parskip}{0pt}}
\setcounter{secnumdepth}{-2}
% Redefines (sub)paragraphs to behave more like sections
\ifx\paragraph\undefined\else
  \let\oldparagraph\paragraph
  \renewcommand{\paragraph}[1]{\oldparagraph{#1}\mbox{}}
\fi
\ifx\subparagraph\undefined\else
  \let\oldsubparagraph\subparagraph
  \renewcommand{\subparagraph}[1]{\oldsubparagraph{#1}\mbox{}}
\fi

% set default figure placement to htbp
\makeatletter
\def\fps@figure{htbp}
\makeatother


\title{Tarea 1}
\author{Pablo,Sofía,Román}
\date{26/1/2020}

\begin{document}
\maketitle

\#Ejercicio 1

\begin{Shaded}
\begin{Highlighting}[]
\NormalTok{  A <-}\StringTok{ }\KeywordTok{matrix}\NormalTok{(}\KeywordTok{c}\NormalTok{(}\DecValTok{7}\NormalTok{,}\DecValTok{5}\NormalTok{,}\DecValTok{3}\NormalTok{,}\DecValTok{2}\NormalTok{,}\DecValTok{1}\NormalTok{,}\DecValTok{8}\NormalTok{),}\DataTypeTok{ncol =} \DecValTok{3}\NormalTok{)}
\NormalTok{  B <-}\StringTok{ }\KeywordTok{matrix}\NormalTok{(}\KeywordTok{c}\NormalTok{(}\DecValTok{11}\NormalTok{,}\OperatorTok{-}\DecValTok{7}\NormalTok{,}\DecValTok{8}\NormalTok{,}\DecValTok{12}\NormalTok{,}\DecValTok{0}\NormalTok{,}\DecValTok{9}\NormalTok{),}\DataTypeTok{ncol =} \DecValTok{3}\NormalTok{)}
  \CommentTok{#a) A'}
    \KeywordTok{t}\NormalTok{(A)}
\end{Highlighting}
\end{Shaded}

\begin{verbatim}
##      [,1] [,2]
## [1,]    7    5
## [2,]    3    2
## [3,]    1    8
\end{verbatim}

\begin{Shaded}
\begin{Highlighting}[]
  \CommentTok{#b) A − B}
\NormalTok{    A }\OperatorTok{-}\StringTok{ }\NormalTok{B}
\end{Highlighting}
\end{Shaded}

\begin{verbatim}
##      [,1] [,2] [,3]
## [1,]   -4   -5    1
## [2,]   12  -10   -1
\end{verbatim}

\begin{Shaded}
\begin{Highlighting}[]
  \CommentTok{#c) AB  }
\NormalTok{    A}\OperatorTok{*}\NormalTok{B}
\end{Highlighting}
\end{Shaded}

\begin{verbatim}
##      [,1] [,2] [,3]
## [1,]   77   24    0
## [2,]  -35   24   72
\end{verbatim}

\begin{Shaded}
\begin{Highlighting}[]
  \CommentTok{#d) A'A}
    \KeywordTok{t}\NormalTok{(A)}\OperatorTok\NormalTok{A}
\end{Highlighting}
\end{Shaded}

\begin{verbatim}
##      [,1] [,2] [,3]
## [1,]   74   31   47
## [2,]   31   13   19
## [3,]   47   19   65
\end{verbatim}

\begin{Shaded}
\begin{Highlighting}[]
  \CommentTok{#e) AA'}
\NormalTok{    A}\OperatorTok\KeywordTok{t}\NormalTok{(A)}
\end{Highlighting}
\end{Shaded}

\begin{verbatim}
##      [,1] [,2]
## [1,]   59   49
## [2,]   49   93
\end{verbatim}

\begin{Shaded}
\begin{Highlighting}[]
  \CommentTok{#f) A + B}
\NormalTok{    A }\OperatorTok{+}\StringTok{ }\NormalTok{B}
\end{Highlighting}
\end{Shaded}

\begin{verbatim}
##      [,1] [,2] [,3]
## [1,]   18   11    1
## [2,]   -2   14   17
\end{verbatim}

\begin{Shaded}
\begin{Highlighting}[]
  \CommentTok{#g) A'B}
    \KeywordTok{t}\NormalTok{(A)}\OperatorTok\NormalTok{B}
\end{Highlighting}
\end{Shaded}

\begin{verbatim}
##      [,1] [,2] [,3]
## [1,]   42  116   45
## [2,]   19   48   18
## [3,]  -45  104   72
\end{verbatim}

\begin{Shaded}
\begin{Highlighting}[]
  \CommentTok{#h) AB'}
\NormalTok{    A}\OperatorTok\KeywordTok{t}\NormalTok{(B)}
\end{Highlighting}
\end{Shaded}

\begin{verbatim}
##      [,1] [,2]
## [1,]  101   -4
## [2,]   71   61
\end{verbatim}

\begin{Shaded}
\begin{Highlighting}[]
  \CommentTok{#i) 17.3*A}
    \FloatTok{17.3}\OperatorTok{*}\NormalTok{A}
\end{Highlighting}
\end{Shaded}

\begin{verbatim}
##       [,1] [,2]  [,3]
## [1,] 121.1 51.9  17.3
## [2,]  86.5 34.6 138.4
\end{verbatim}

\begin{Shaded}
\begin{Highlighting}[]
  \CommentTok{#j) (1/19)*B}
\NormalTok{    (}\DecValTok{1}\OperatorTok{/}\DecValTok{19}\NormalTok{)}\OperatorTok{*}\NormalTok{B}
\end{Highlighting}
\end{Shaded}

\begin{verbatim}
##            [,1]      [,2]      [,3]
## [1,]  0.5789474 0.4210526 0.0000000
## [2,] -0.3684211 0.6315789 0.4736842
\end{verbatim}

\hypertarget{ejercicio-3}{%
\section{Ejercicio 3}\label{ejercicio-3}}

\hypertarget{las-flores-de-fisher-y-anderson}{%
\subsection{Las flores de Fisher y
Anderson}\label{las-flores-de-fisher-y-anderson}}

\hypertarget{a}{%
\subsubsection{3a}\label{a}}

\begin{Shaded}
\begin{Highlighting}[]
\NormalTok{X <-}\StringTok{ }\NormalTok{iris3[,,}\DecValTok{1}\NormalTok{] }\CommentTok{#1 is for Setosa}
\NormalTok{s_mn <-}\StringTok{ }\KeywordTok{apply}\NormalTok{(}\DataTypeTok{X =}\NormalTok{ X,}\DataTypeTok{MARGIN =} \DecValTok{2}\NormalTok{,}\DataTypeTok{FUN =}\NormalTok{ mean)}

\CommentTok{#corrected mean square }
\NormalTok{dim_set <-}\StringTok{ }\KeywordTok{dim}\NormalTok{(X)[}\DecValTok{1}\NormalTok{]}
\NormalTok{s_mn_matrix <-}\StringTok{ }\KeywordTok{matrix}\NormalTok{(}\KeywordTok{rep}\NormalTok{(s_mn,}\DataTypeTok{each =}\NormalTok{ dim_set), }\DataTypeTok{nrow =}\NormalTok{ dim_set)}
\NormalTok{A <-}\StringTok{ }\NormalTok{X }\OperatorTok{-}\StringTok{ }\NormalTok{s_mn_matrix}
\CommentTok{# for(i in 1:4)\{}
\CommentTok{#   A[,i] <- A[,i] * A[,i]}
\CommentTok{# \}}
\NormalTok{A <-}\StringTok{ }\KeywordTok{t}\NormalTok{(A) }\OperatorTok\StringTok{ }\NormalTok{A}
\CommentTok{#unbaised sample covarianse}
\NormalTok{Sx <-}\StringTok{ }\NormalTok{(}\DecValTok{1}\OperatorTok{/}\NormalTok{(dim_set }\DecValTok{-1}\NormalTok{) }\OperatorTok{*}\StringTok{ }\NormalTok{A)}
\end{Highlighting}
\end{Shaded}

\hypertarget{b}{%
\subsubsection{3b}\label{b}}

\begin{Shaded}
\begin{Highlighting}[]
\CommentTok{#eigenpar }
\NormalTok{eigen_list <-}\StringTok{ }\KeywordTok{eigen}\NormalTok{(Sx)}
\end{Highlighting}
\end{Shaded}

\hypertarget{c}{%
\subsubsection{3c}\label{c}}

\begin{Shaded}
\begin{Highlighting}[]
\NormalTok{U <-}\StringTok{ }\NormalTok{eigen_list}\OperatorTok{$}\NormalTok{vectors}
\NormalTok{L <-}\StringTok{ }\KeywordTok{diag}\NormalTok{(eigen_list}\OperatorTok{$}\NormalTok{values)}

\CommentTok{#compute Sx = ULU'}
\NormalTok{Sx_prim =}\StringTok{ }\NormalTok{U }\OperatorTok\StringTok{ }\NormalTok{L }\OperatorTok\StringTok{ }\KeywordTok{t}\NormalTok{(U)}

\CommentTok{#compute UU' & U'U }
\NormalTok{UUt <-}\StringTok{ }\NormalTok{U }\OperatorTok\StringTok{ }\KeywordTok{t}\NormalTok{(U)}
\NormalTok{UtU <-}\StringTok{ }\KeywordTok{t}\NormalTok{(U) }\OperatorTok\StringTok{ }\NormalTok{U}
\end{Highlighting}
\end{Shaded}

\#3d

\begin{Shaded}
\begin{Highlighting}[]
\NormalTok{matplot_setosa <-}\StringTok{ }\NormalTok{iris }\OperatorTok\StringTok{ }
\StringTok{                  }\KeywordTok{filter}\NormalTok{(Species }\OperatorTok{==}\StringTok{ "setosa"}\NormalTok{) }\OperatorTok\StringTok{  }
\StringTok{                  }\KeywordTok{select}\NormalTok{(Sepal.Length, Sepal.Width, Petal.Length) }\OperatorTok
\StringTok{                  }\KeywordTok{ggpairs}\NormalTok{() }\OperatorTok{+}
\StringTok{                  }\KeywordTok{theme_bw}\NormalTok{() }\OperatorTok{+}
\StringTok{                  }\KeywordTok{labs}\NormalTok{(}\DataTypeTok{title =} \StringTok{"SETOSA"}\NormalTok{, }\DataTypeTok{x =} \StringTok{""}\NormalTok{, }\DataTypeTok{y =} \StringTok{""}\NormalTok{)}

\NormalTok{matplot_veris <-}\StringTok{ }\NormalTok{iris }\OperatorTok\StringTok{ }
\StringTok{                  }\KeywordTok{filter}\NormalTok{(Species }\OperatorTok{==}\StringTok{ "versicolor"}\NormalTok{) }\OperatorTok\StringTok{  }
\StringTok{                  }\KeywordTok{select}\NormalTok{(Sepal.Length, Sepal.Width, Petal.Length) }\OperatorTok
\StringTok{                  }\KeywordTok{ggpairs}\NormalTok{() }\OperatorTok{+}\StringTok{ }
\StringTok{                  }\KeywordTok{theme_bw}\NormalTok{() }\OperatorTok{+}
\StringTok{                  }\KeywordTok{labs}\NormalTok{(}\DataTypeTok{title =} \StringTok{"VERSICOLOR"}\NormalTok{, }\DataTypeTok{x =} \StringTok{""}\NormalTok{, }\DataTypeTok{y =} \StringTok{""}\NormalTok{)}

\NormalTok{matplot_virg <-}\StringTok{ }\NormalTok{iris }\OperatorTok\StringTok{ }
\StringTok{                  }\KeywordTok{filter}\NormalTok{(Species }\OperatorTok{==}\StringTok{ "virginica"}\NormalTok{) }\OperatorTok\StringTok{  }
\StringTok{                  }\KeywordTok{select}\NormalTok{(Sepal.Length, Sepal.Width, Petal.Length) }\OperatorTok
\StringTok{                  }\KeywordTok{ggpairs}\NormalTok{() }\OperatorTok{+}
\StringTok{                  }\KeywordTok{theme_bw}\NormalTok{() }\OperatorTok{+}
\StringTok{                  }\KeywordTok{labs}\NormalTok{(}\DataTypeTok{title =} \StringTok{"VIRGINICA"}\NormalTok{, }\DataTypeTok{x =} \StringTok{""}\NormalTok{, }\DataTypeTok{y =} \StringTok{""}\NormalTok{)}


\NormalTok{matplot_all <-}\StringTok{ }\NormalTok{iris }\OperatorTok\StringTok{ }
\StringTok{                  }\KeywordTok{ggscatmat}\NormalTok{(}\DataTypeTok{color =} \StringTok{'Species'}\NormalTok{) }\OperatorTok{+}\StringTok{ }
\StringTok{                  }\KeywordTok{theme_bw}\NormalTok{() }\OperatorTok{+}\StringTok{ }
\StringTok{                  }\KeywordTok{labs}\NormalTok{(}\DataTypeTok{title =} \StringTok{"FLORES"}\NormalTok{, }\DataTypeTok{x =} \StringTok{""}\NormalTok{, }\DataTypeTok{y =} \StringTok{""}\NormalTok{, }\DataTypeTok{color =} \StringTok{"Especie de}\CharTok{\textbackslash{}n}\StringTok{ flor"}\NormalTok{)}
\end{Highlighting}
\end{Shaded}

\begin{verbatim}
## Warning in ggscatmat(., color = "Species"): Factor variables are omitted in plot
\end{verbatim}

\begin{Shaded}
\begin{Highlighting}[]
\CommentTok{#we omitted this graph because it looked very heaped}
\CommentTok{# matplot_all2 <- iris %>% }
\CommentTok{#                   ggpairs(mapping = aes(color = 'Species')) + }
\CommentTok{#                   theme_light() + }
\CommentTok{#                   labs(title = "Figura 1", x = "", y = "", color = "Especie de\textbackslash{}n flor")}


\NormalTok{matplot_setosa}
\end{Highlighting}
\end{Shaded}

\includegraphics{hw1_files/figure-latex/unnamed-chunk-5-1.pdf}

\begin{Shaded}
\begin{Highlighting}[]
\NormalTok{matplot_veris}
\end{Highlighting}
\end{Shaded}

\includegraphics{hw1_files/figure-latex/unnamed-chunk-5-2.pdf}

\begin{Shaded}
\begin{Highlighting}[]
\NormalTok{matplot_virg}
\end{Highlighting}
\end{Shaded}

\includegraphics{hw1_files/figure-latex/unnamed-chunk-5-3.pdf}

\begin{Shaded}
\begin{Highlighting}[]
\NormalTok{matplot_all}
\end{Highlighting}
\end{Shaded}

\includegraphics{hw1_files/figure-latex/unnamed-chunk-5-4.pdf}

\hypertarget{ejercicio-4}{%
\section{Ejercicio 4}\label{ejercicio-4}}

\hypertarget{flores-de-fisher-y-anderson-parte-ii.}{%
\subsection{Flores de Fisher y Anderson parte
II.}\label{flores-de-fisher-y-anderson-parte-ii.}}

\hypertarget{a-1}{%
\subsubsection{4a}\label{a-1}}

Sabemos que \(Y^{5} = X^{3} + X^{4}.\) Entonces, para encontar \(C\) tal
que \[ Y = XC,\] se puede notar que \(C\) debe ser de la forma

{[} C= \left[
\begin{array}{c|c}
I_{4,4} & 

\begin{array}{c}
0 \\ 0 \\ 1 \\ 1\end{array}


\end{array}
\right]{]}.

\begin{Shaded}
\begin{Highlighting}[]
\NormalTok{Y <-}\StringTok{ }\KeywordTok{cbind}\NormalTok{(X,(X[,}\StringTok{'Petal L.'}\NormalTok{] }\OperatorTok{+}\StringTok{ }\NormalTok{X[,}\StringTok{'Petal W.'}\NormalTok{]))}
\KeywordTok{colnames}\NormalTok{(Y) <-}\StringTok{ }\KeywordTok{c}\NormalTok{(}\StringTok{'Sepal L.'}\NormalTok{,}\StringTok{'Sepal W.'}\NormalTok{,}\StringTok{'Petal L.'}\NormalTok{, }\StringTok{'Petal W.'}\NormalTok{, }\StringTok{'PL + PW'}\NormalTok{)}

\CommentTok{#C}
\NormalTok{C <-}\StringTok{  }\KeywordTok{diag}\NormalTok{(}\DataTypeTok{x =} \DecValTok{1}\NormalTok{, }\DataTypeTok{nrow =} \DecValTok{4}\NormalTok{)}
\NormalTok{C <-}\StringTok{  }\KeywordTok{cbind}\NormalTok{(C,}\KeywordTok{c}\NormalTok{(}\DecValTok{0}\NormalTok{,}\DecValTok{0}\NormalTok{,}\DecValTok{1}\NormalTok{,}\DecValTok{1}\NormalTok{))}

\CommentTok{#check if Y = XC}
\NormalTok{testY <-}\StringTok{ }\NormalTok{Y }\OperatorTok{==}\StringTok{ }\NormalTok{X }\OperatorTok\StringTok{ }\NormalTok{C}
\end{Highlighting}
\end{Shaded}

\hypertarget{b-1}{%
\subsubsection{4b}\label{b-1}}

\begin{Shaded}
\begin{Highlighting}[]
\CommentTok{#covariance matrix}
\NormalTok{dim_Y <-}\StringTok{ }\KeywordTok{dim}\NormalTok{(Y)[}\DecValTok{1}\NormalTok{]}
\NormalTok{Sy <-}\StringTok{ }\NormalTok{(}\DecValTok{1}\OperatorTok{/}\NormalTok{(dim_Y }\DecValTok{-1}\NormalTok{)) }\OperatorTok{*}\StringTok{ }\KeywordTok{t}\NormalTok{(Y) }\OperatorTok\StringTok{ }\NormalTok{Y}

\CommentTok{#eigenpair}
\NormalTok{eigen_listY <-}\StringTok{ }\KeywordTok{eigen}\NormalTok{(Sy)}
\end{Highlighting}
\end{Shaded}

\hypertarget{c-1}{%
\subsubsection{4c}\label{c-1}}

\begin{Shaded}
\begin{Highlighting}[]
\NormalTok{testSy <-}\StringTok{ }\KeywordTok{t}\NormalTok{(C) }\OperatorTok\StringTok{ }\NormalTok{Sx }\OperatorTok\StringTok{ }\NormalTok{C }
\end{Highlighting}
\end{Shaded}

\hypertarget{ejercicio-5}{%
\section{Ejercicio 5}\label{ejercicio-5}}

\hypertarget{eda-de-los-indicadores-de-la-cnbv}{%
\subsection{EDA de los Indicadores de la
CNBV}\label{eda-de-los-indicadores-de-la-cnbv}}

\begin{Shaded}
\begin{Highlighting}[]
\NormalTok{doc <-}\StringTok{ "DatosCNBVModificados.csv"}
\NormalTok{data_cnbv <-}\StringTok{ }\KeywordTok{read.csv}\NormalTok{(}\DataTypeTok{file =}\NormalTok{ doc)}
\end{Highlighting}
\end{Shaded}

\end{document}
